

\section{Tutorial Problem}
In the following section, we shall provide an example cook book for preparing and running a~model,
based on one of the test problems, namely
\begin{verbatim}
tests/21_solute_fv_frac/03_fv_dp_sorp_small.yaml.
\end{verbatim}
We shall start with the preparation of the geometry using an external software and then we shall go step by step through the 
commented main input file. The problem includes steady Darcy flow, transport of two substances with explicit
time discretization and a~reaction term consisting of dual porosity and sorption model.
Further tutorials focused on particular features can be found in Chapter \ref{chapter:tutorials}.

\subsection{Geometry}
We consider a~simple 2D problem with a~branching 1D fracture (see Figure \ref{fig:tutorial} for the geometry). 
To prepare a~mesh file we use the \href{http://geuz.org/gmsh/}{GMSH software}.
First, we construct a~geometry file. In our case the geometry consists of: 
\begin{itemize}
 \item one physical 2D domain corresponding to the whole square
 \item three 1D physical domains of the fracture
 \item four 1D boundary physical domains of the 2D domain
 \item three 0D boundary physical domains of the 1D domain
\end{itemize}
In this simple example, we can in fact combine physical domains in every group, however we use this more complex setting for
demonstration purposes. Using GMSH graphical interface we can prepare the GEO file where physical domains are referenced by numbers, then we use 
any text editor and replace numbers with string labels in such a~way that the labels of boundary physical domains start with the dot character. 
These are the domains where we will not do any calculations but we will use them for setting boundary conditions.
Finally, we get the GEO file like this:

\begin{multicols}{2}
{\small
\begin{Verbatim}[numbers=left]
cl1 = 0.16;
Point(1) = {0, 1, 0, cl1};
Point(2) = {1, 1, 0, cl1};
Point(3) = {1, 0, 0, cl1};
Point(4) = {0, 0, 0, cl1};
Point(6) = {0.25, -0, 0, cl1};
Point(7) = {0, 0.25, 0, cl1};
Point(8) = {0.5, 0.5, -0, cl1};
Point(9) = {0.75, 1, 0, cl1};
Line(19) = {9, 8};
Line(20) = {7, 8};
Line(21) = {8, 6};
Line(22) = {2, 3};
Line(23) = {2, 9};
Line(24) = {9, 1};
Line(25) = {1, 7};
Line(26) = {7, 4};
Line(27) = {4, 6};
Line(28) = {6, 3};
\end{Verbatim}
\columnbreak
\begin{Verbatim}[numbers=left, firstnumber=last]
Line Loop(30) = {20, -19, 24, 25};
Plane Surface(30) = {30};
Line Loop(32) = {23, 19, 21, 28, -22};
Plane Surface(32) = {32};
Line Loop(34) = {26, 27, -21, -20};
Plane Surface(34) = {34};
Physical Point(".1d_top") = {9};
Physical Point(".1d_left") = {7};
Physical Point(".1d_bottom") = {6};
Physical Line("1d_upper") = {19};
Physical Line("1d_lower") = {21};
Physical Line("1d_left_branch") = {20};
Physical Line(".2d_top") = {23, 24};
Physical Line(".2d_right") = {22};
Physical Line(".2d_bottom") = {27, 28};
Physical Line(".2d_left") = {25, 26};
Physical Surface("2d") = {30, 32, 34};
\end{Verbatim}
}
\end{multicols}

Notice the labeled physical domains on lines 26 -- 36. Then we just set the discretization step \verb'cl1' and use GMSH to create the mesh file.
The mesh file contains both the 'bulk' elements where we perform calculations and the 'boundary' elements (on the boundary physical domains) where we only set the boundary conditions.

\subsection{YAML File Format}
The main input file uses the YAML file format with some restrictions. 
We prefer to call YAML objects ``records'' and we introduce also ``abstract records''
that mimic C++ abstract classes.
Arrays have only elements of the same type (possibly using abstract record types for polymorphism). 
The usual keys are in lower case and without spaces (using underscores instead).
For detailed description see Section \ref{sec:CONformat}.


Having the computational mesh from the previous step, we can create the main input file with the description of our problem. 
\begin{Verbatim}[numbers=left]
flow123d_version: 3.1.0
problem: !CouplingSequential
  description: Tutorial problem: Transport 1D-2D (convection, dual
               porosity, sorption, sources).
  mesh:
    mesh_file: ../00_mesh/square_1x1_frac_fork.msh
    regions:
      - !Union
        name: 1d_domain
        regions:
          - 1d_upper
          - 1d_lower
          - 1d_left_branch
\end{Verbatim}
The file starts with a~selection of problem type (\verb'CouplingSequential'), and a~textual problem description.
Next, we specify the computational mesh, here it consists of the name of the mesh file and the declaration of one region 
given as the union of all 1D regions i.e. representing the whole fracture. Other keys of the \Alink{IT::Mesh}{\tt mesh} record allow labeling regions given only by numbers, 
defining new regions in terms of element numbers (e.g to have leakage on single element), 
defining boundary regions, and several operations with region sets, see Section \ref{sec:Mesh} for details.

\subsection{Flow Setting}
Next, we setup the flow problem. We shall consider a~steady flow (i.e. with zero storativity) driven only by the pressure gradient (no gravity),
setting the Dirichlet boundary condition on the whole boundary with the pressure head equal to $x+y$. 
The \Alink{Flow-Darcy-MH-Data::conductivity}{\tt conductivity} will be $k_2=10^{-7}$ \unitss{}{1}{-1} on the 2D domain and $k_1=10^{-6}$ \unitss{}{1}{-1} on the 1D domain.
We leave the default value for the \Alink{Flow-Darcy-MH-Data::cross-section}{\tt cross\_section} of the 2D domain,
meaning that the thickness of 2D domain is $\delta_2=1$ \unitss{}{1}{}.
For the 1D fractures cross-section, we prescribe $\delta_1=0.04$ \unitss{}{2}{} on the 1D domain.
The transition coefficient $\sigma_2$ between dimensions can be scaled by setting the dimensionless parameter 
$\sigma_{21}$ (\Alink{Flow-Darcy-MH-Data::sigma}{\tt sigma}). This can be used for simulating additional
effects which prevent the liquid transition from/to a~fracture, like a~thin resistance layer.
Notice that the scaling parameter is set on the lower dimensional domain, i.e. $\sigma_{21}=0.9$ on line \ref{vrb:tutorial_flow_input_sigma}.
Read Section \ref{sec:darcy_flow} for more details.

\begin{Verbatim}[numbers=left, firstnumber=last,commandchars=\\\{\}]
  flow_equation: !Flow_Darcy_MH\label{vrb:tutorial_floweq}
    input_fields:\label{vrb:tutorial_flow_input_start}
      - region: 1d_domain
        conductivity: 1.0e-06
        cross_section: 0.04
        sigma: 0.9\label{vrb:tutorial_flow_input_sigma}
      - region: 2d
        conductivity: 1.0e-07
      - region: .BOUNDARY
        bc_type: dirichlet
        bc_pressure: !FieldFormula
          value: x+y\label{vrb:tutorial_flow_input_end}
    nonlinear_solver:
      linear_solver: !Petsc
        a_tol: 1.0e-12
        r_tol: 1.0e-12
    output:\label{vrb:tutorial_output_start}
      fields:
        - pressure_p0
        - velocity_p0\label{vrb:tutorial_output_end}
    output_stream:
      file: flow.pvd
      format: !vtk
        variant: ascii
\end{Verbatim}
% Tutorial also has output of pressure_p1 is it still supported?
%
On line \ref{vrb:tutorial_floweq}, we specify particular implementation (numerical method) of the flow solver, in this case the Mixed-Hybrid
solver for steady problems. On lines \ref{vrb:tutorial_flow_input_start} -- \ref{vrb:tutorial_flow_input_end} in the array \hyperA{IT::Flow-Darcy-MH-Data}{\tt input\_fields}, 
we set both mathematical fields that live on the computational domain 
and those defining the boundary conditions.
We use implicitly defined region ``.BOUNDARY'' that contains all boundary regions and we set there Dirichlet boundary condition in terms of the 
pressure head. In this case, the field is not of the implicit type {\tt FieldConstant}, so we must specify the type of the field {\tt !FieldFormula}.
See Section \ref{sec:Fields} for other field types. 
Next, we specify the type of the linear solver and its tolerances.
On lines \ref{vrb:tutorial_output_start} -- \ref{vrb:tutorial_output_end}, we specify which output fields should be written to the output stream
(that means particular output file, with given format). See Section \ref{section_output} for the list of available output \Alink{IT::Flow-Darcy-MH-OutputFields}{\tt fields}.
Currently, we support only one \Alink{IT::OutputStream}{\tt output\_stream} per equation. We specify the filename and the format of the output stream (the used ASCII VTK format is the default).



\subsection{Transport Setting}
The flow model is followed by a~transport model in the record \Alink{Coupling-Sequential::solute-equation}{\tt solute\_equation}
beginning on line \ref{vrb:tutorial_solute_start}. Here, we use an implementation called \hyperA{IT::Solute-Advection-FV}{Solute\_Advection\_FV}
which stands for an explicit finite volume solver of the convection equation (without diffusion).
The operator splitting method is used for equilibrium sorption as well as for dual porosity model and 
first order reactions simulation.

\begin{Verbatim}[numbers=left, firstnumber=last,commandchars=\\\{\}]
  solute_equation: !Coupling_OperatorSplitting\label{vrb:tutorial_solute_start}
    substances:\label{vrb:tutorial_substances_start}
      - name: age # water age
        molar_mass: 0.018
      - name: U235 # uranium 235
        molar_mass: 0.235\label{vrb:tutorial_substances_end}
    transport: !Solute_Advection_FV
      input_fields:\label{vrb:tutorial_transport_input_start}
        - region: ALL
          init_conc: 0\label{vrb:tutorial_transport_init_conc}
          porosity: 0.25
          # source is in the whole volume (l+s) -> times porosity
          sources_density:
            - 0.25
            - 0
        - region: .BOUNDARY\label{vrb:tutorial_transport_boundary}
          bc_conc:
            - 0.0
            - 1.0\label{vrb:tutorial_transport_input_end}
    time:
      end_time: 1000000
    balance:
      cumulative: true
\end{Verbatim}

On lines \ref{vrb:tutorial_substances_start} -- \ref{vrb:tutorial_substances_end}, we set the transported \Alink{Coupling-OperatorSplitting::substances}{\tt substances}, 
which are identified by their names. Here, the first one is the \verb'age' of the water, with the molar mass of water, 
and the second one \verb'U235' is the uranium isotope 235. 
On lines \ref{vrb:tutorial_transport_input_start} -- \ref{vrb:tutorial_transport_input_end}, we set the input fields, in particular zero initial concentration for all substances,
\Alink{Solute-Advection-FV-Data::porosity}{\tt porosity} $\theta = 0.25$ and 
sources of concentration by \Alink{Solute-Advection-FV-Data::sources-density}{\tt sources\_density}. 
Notice line \ref{vrb:tutorial_transport_init_conc} where we can see only single value since an automatic conversion is applied to turn the scalar 
zero into the zero vector (of size 2 according to the number of substances). 

The boundary fields are set on lines \ref{vrb:tutorial_transport_boundary} -- \ref{vrb:tutorial_transport_input_end}. We do not need to specify the type of the condition since there is 
only one type in the {\tt Solute\_Advection\_FV} transport model. The boundary condition is equal to $1$ for the uranium 235 and $0$ 
for the age of the water and is automatically applied only on the inflow part of the boundary. 

We also have to prescribe the \hyperA{IT::TimeGovernor}{\tt time} setting -- here it is only the end time of the simulation
(in seconds: $10^6\,\rm{s}\approx 11.57$ days). The step size is determined automatically from the CFL condition,
however, a~smaller time step can be enforced if necessary.

Reaction term of the transport model is described in the next subsection, including dual porosity and sorption.

\subsection{Reaction Term}\label{subsubsec:reactions}
The input information for dual porosity, equilibrial sorption and possibly first order reactions are enclosed in the record 
\Alink{Coupling-OperatorSplitting::reaction-term}{\tt reaction\_term}, lines \ref{vrb:tutorial_dp_start} -- \ref{vrb:tutorial_dp_end}. Go to section \ref{sec:reaction_term}
to see how the models can be chained.

The type of the first process is determined by {\tt !DualPorosity}, on line \ref{vrb:tutorial_dp_start}. 
The \Alink{DualPorosity::input-fields}{\tt input\_fields}
of the dual porosity model are set on lines \ref{vrb:tutorial_dp_input_start} -- \ref{vrb:tutorial_dp_input_end} and the output is disabled by setting an empty array on line \ref{vrb:tutorial_dp_output}.

\begin{Verbatim}[numbers=left, firstnumber=last,commandchars=\\\{\}]
    reaction_term: !DualPorosity\label{vrb:tutorial_dp_start}
      input_fields:\label{vrb:tutorial_dp_input_start}
        - region: ALL
          diffusion_rate_immobile:
            - 0.01
            - 0.01
          porosity_immobile: 0.25
          init_conc_immobile:
            - 0.0
            - 0.0\label{vrb:tutorial_dp_input_end}
      output:
        fields: []\label{vrb:tutorial_dp_output}
      reaction_mobile: !SorptionMobile\label{vrb:tutorial_sorp_mob}
        solvent_density: 1000.0 # water\label{vrb:tutorial_sorp_mob_param_start}
        substances:
          - age
          - U235
        solubility:
          - 1.0
          - 1.0\label{vrb:tutorial_sorp_mob_param_end}
        input_fields: &anchor1\label{vrb:tutorial_sorp_mob_input_start}
          - region: ALL
            rock_density: 2800.0 # granite
            sorption_type:
              - none
              - freundlich
            distribution_coefficient:
              - 0
              - 1.598e-4
            isotherm_other:
              - 0
              - 1.0\label{vrb:tutorial_sorp_mob_input_end}
        output:
          fields: []\label{vrb:tutorial_sorp_mob_output}
      reaction_immobile: !SorptionImmobile\label{vrb:tutorial_sorp_immob}
        solvent_density: 1000.0 # water
        substances:
          - age
          - U235
        solubility:
          - 1.0
          - 1.0
        input_fields: *anchor1\label{vrb:tutorial_sorp_immob_input}
        output:
          fields: []\label{vrb:tutorial_dp_end}
    output_stream:\label{vrb:tutorial_trans_output_start}
      file: transport.pvd
      format: !vtk
        variant: ascii
      times:
        - step: 100000.0\label{vrb:tutorial_trans_output_end}
\end{Verbatim}

Next, we define the equilibrial sorption model such that \hyperA{IT::SorptionMobile}{\tt SorptionMobile} type takes place in the mobile 
zone of the dual porosity model while \hyperA{IT::SorptionImmobile}{\tt SorptionImmobile} type takes place in its immobile zone, see lines \ref{vrb:tutorial_sorp_mob} and \ref{vrb:tutorial_sorp_immob}.
Isothermally described sorption simulation can be used in the case of low concentrated solutions without competition between multiple dissolved species.

On lines \ref{vrb:tutorial_sorp_mob_param_start} -- \ref{vrb:tutorial_sorp_mob_param_end}, we set the sorption related input information. The solvent is water so the \Alink{Sorption::solvent-density}{\tt solvent\_density} 
is supposed to be constant all over the simulated area. The vector \Alink{Sorption::substances}{\tt substances} 
contains the list of names of solute substances which are considered to be affected by the sorption.
Solubility is a~material characteristic of a~sorbing substance related to the solvent. Elements of the vector 
\Alink{Sorption::solubility}{\tt solubility} define the upper bound of aqueous concentration which can appear.
This constrain is necessary because some substances might have limited solubility and if the solubility exceeds 
its limit they start to precipitate. {\tt solubility} is a~crucial parameter for solving a~set of nonlinear 
equations, described further. 

The record \hyperA{IT::Sorption-Data}{\tt input\_fields} covers the region specific parameters.
All implemented types of sorption can take the rock density in the region into account. The value of 
\Alink{Sorption-Data::rock-density}{\tt rock\_density} is a~constant in our case. 
The \Alink{Sorption-Data::sorption-type}{\tt sorption\_type} represents the empirically determined isotherm 
type and can have one of four possible values: \{{\tt"none"}, {\tt"linear"}, {\tt"freundlich"}, {\tt"langmuir"}\}. 
Linear isotherm needs just one parameter given whereas Freundlich and Langmuir isotherms require two parameters. 
We will use Freundlich isotherm for demonstration but we will set the other parameter (exponent) $\alpha=1$ 
which means it will be the same as the linear type. We use value 
$K_d=1.598\cdot10^{-4}$ \unitss{-1}{3}{} for the 
\Alink{Sorption-Data::distribution-coefficient}{\tt distribution\_coefficient} 
according to (www.skb.se, report R-10-48 by James Crawford, 2010).
 

On line \ref{vrb:tutorial_sorp_immob_input}, notice the reference pointing to the definition of input fields on lines \ref{vrb:tutorial_sorp_mob_input_start} -- \ref{vrb:tutorial_sorp_mob_input_end}. Only entire records 
can be referenced which is why we have to repeat parts of the input such as solvent density and solubility 
(records for reaction mobile and reaction immobile have different types).

On lines \ref{vrb:tutorial_sorp_mob_output} and \ref{vrb:tutorial_dp_end}, we define which sorption specific outputs are to be written to the output file. 
An implicit set of outputs exists. In this case we define an empty set of outputs thus overriding the implicit one. 
This means that no sorption specific outputs will be written to the output file.
On lines \ref{vrb:tutorial_trans_output_start} -- \ref{vrb:tutorial_trans_output_end} we specify which output fields should be written to the output stream. Currently, we support output into VTK and GMSH data format.
In the output record for time-dependent process we have to specify the time {\tt step} (line \ref{vrb:tutorial_trans_output_end}) which determines the frequency of output data writing.



\subsection{Results}
In Figure \ref{fig:tutorial} one can see the results: the pressure and the velocity field on the left and the 
concentration of U235 at time $t=9\cdot10^{5}$ s on the right. Even though the pressure gradient is the same both in the 
2D domain and in the fracture, the velocity field is ten times faster in the fracture due to its higher conductivity. 
Since porosity is the same, the substance is transported faster by the fracture. Therefore nonzero concentration appears in the bottom 
left of the 2D domain long before the main wave propagating solely through the 2D domain reaches that corner.


\begin{figure}[!ht]
    \centering
    \begin{subfigure}[b]{0.48\textwidth}
        \centering
        \includegraphics[width=\textwidth]{\fig/03_flow.pdf}
        % 03_flow.pdf: no raster
        \caption{Elementwise pressure head and\\velocity field denoted by triangles.\\ (Steady flow.)}
        \label{fig:tut-flow}
    \end{subfigure}
    ~
    \begin{subfigure}[b]{0.48\textwidth}
        \centering
        \includegraphics[width=\textwidth]{\fig/03_trans.pdf}
        % 03_trans.pdf: no raster
        \caption{Propagation of U235 from the inflow part of the boundary. \\ (At the time $9\cdot10^{5}$ s.)}
        \label{fig:tut-trans}
    \end{subfigure}
    \caption{Results of the tutorial problem.}
    \label{fig:tutorial}
\end{figure}
