% Copyright (C) 2007 Technical University of Liberec.  All rights reserved.
%
% Please make a following refer to Flow123d on your project site if you use the program for any purpose,
% especially for academic research:
% Flow123d, Research Centre: Advanced Remedial Technologies, Technical University of Liberec, Czech Republic
%
% This program is free software; you can redistribute it and/or modify it under the terms
% of the GNU General Public License version 3 as published by the Free Software Foundation.
%
% This program is distributed in the hope that it will be useful, but WITHOUT ANY WARRANTY;
% without even the implied warranty of MERCHANTABILITY or FITNESS FOR A PARTICULAR PURPOSE.
% See the GNU General Public License for more details.
%
% You should have received a copy of the GNU General Public License along with this program; if not,
% write to the Free Software Foundation, Inc., 59 Temple Place - Suite 330, Boston, MA 021110-1307, USA.


\section{Mesh and Data File Format MSH ASCII}
\label{mesh_file}

Currently, the only supported format for the computational mesh is MSH ASCII format in version 2 used
by the GMSH software described as a 
\href{http://gmsh.info//doc/texinfo/gmsh.html#MSH-file-format-version-2-_0028Legacy_0029}{legacy format}.
Same format can be used for the FieldFE data. Support for the new version 4 format is planed for the version 4.0.0 of Flow123d.

The scheme of the file is as follows:
\begin{verbatim}
$MeshFormat
<format version>
$EndMeshFormat

$PhysicalNames
<number of items>
<dimension>     <region ID>     <region label>
...
$EndPhysicalNames

$Nodes
<number of nodes>
<node ID> <X coord> <Y coord> <Z coord>
...
$EndNodes

$Elements
<number of elements>
<element ID> <element shape> <n of tags> <tags> <nodes>
...
$EndElements

$ElementData
<n of string tags>
    <field name>
    <interpolation scheme>
<n of double tags>
    <time>
<n of integer tags>
    <time step index>
    <n of components>
    <n of items>
    <partition index>
<element ID> <component 1> <component 2> ...
...
$EndElementData
\end{verbatim}
Detailed description of individual sections:
\begin{description}
 \item[{\tt PhysicalNames}] : Assign labels to region IDs. Elements of one region should have common dimension. 
    Flow123d interprets regions with labels starting with period as the boundary elements that are not used for calculations.
 \item[{\tt Nodes}] : {\tt <number of nodes>} is also number of data lines that follows. 
    Node IDs are unique but need not to form an arithmetic sequence. Coordinates are float numbers.
 \item[{\tt Elements}] : Element IDs are unique but need not to form an arithmetic sequence. 
    Integer code {\tt <element shape>} represents the shape of element, we support only points (15), lines (1), triangles (2), and tetrahedrons (4).
    Default number of tags is 3. The first is the region ID, the second is ID of the geometrical entity (that was used in original geometry file from which the mesh was generated),
    and the third tag is the partition number. {\tt nodes} is list of node IDs with size according to the element shape.
 \item[{\tt ElementData}] : The header has 2 string tags, 1 double tag, and 4 integer tags with default meaning. For the purpose of the \verb'FieldElementwise' the tags
    \verb'<field name>', \verb'<n of components>', and \verb'<n of items>' are obligatory. This header is followed by field data on individual elements. 
    Flow123d assumes that elements are sorted by {\tt element ID}, but doesn't need to form a~continuous sequence.
\end{description}


%%%%%%%%%%%%%%%%%%%%%%%%%%%%%%%%%%%%%%%%%%%%%%%%%%%%%%%%%%%%%%%%%%%%%%%%%%%%%%%%%%%%%%%%%%%%%
