
In this chapter, we shall describe structure of the main input file and data formats of other input files.
In particular, we briefly describe the GMSH file format used for both the computational mesh as well as for the input of general spatial data.
We only discuss some of the constructs used in the main input file. For details see Chapter \ref{chapter:input-tree-reference} with the reference 
documentation generated from the source files. Alternatively the interactive HTML variant of the reference is in the directory \verb|doc/htmldoc| of 
the installation or at the simulator \href{https://flow.nti.tul.cz/packages/3.9.0_release/htmldoc/}{web page}.


\section{Main Input File}
\label{sec:CONformat}

In this section, we shall describe structure of the main input file that is given either as the first positional argument or through 
the parameter \verb'-s' on the command line. First, we provide a~quick introduction to the YAML file format. Then, we demonstrate the most important 
input structures on the examples. 







\subsection{YAML basics}
YAML is a~human readable data format. It is designed to be both human readable and human editable. As it is not a~markup languages, there are
no tags to determine type of the data. The indentation and justification is used instead for data organization. Moreover the used YAML format (version 1.2) is 
superset of the JSON format, another minimalist data serialization format where braces and brackets are used instead of indentation.
For the more detailed description see \href{https://en.wikipedia.org/wiki/YAML}{Wikipedia} 
for further technical details and for reference parsers for various programming languages see YAML \href{http://yaml.org/}{home page} .

\subsubsection{Hierarchy of Mappings and Lists}
Elementary data are organized to lists and mappings. Let us start with an example of a~list:
\begin{verbatim}
# Example of list 
- 3.14                  # a number
- 2014-01-14            # a date
- Simple string.        # a string
- "3 is three"          # quoting necessary
- '3 is three'          # other quoting
- true                  # boolean
\end{verbatim}
Comments are started by a~hash (\verb'#') which can appear anywhere on a~line and marks the comment up to the end of line.
The the list follows with single item per line preceded by a~dash (\verb'-'). Any value starting by a~digit is interpreted as a~number
or date. Values starting with letter is interpreted as a~string. Otherwise one may use double (\verb'""') or single (\verb"''") 
quotas to mark a~string value explicitly. Finally some strings are interpreted as Boolean values, it is recommended to use 
\verb'true' and \verb'false' (other possible pairs are e.g. \verb'yes/no', \verb'y/n', \verb'on/off'). 

Alternatively a~list may be written in compact "JSON" way enclosing the list into brackets:
\begin{verbatim}
# Compact list
[ 3.14, 2014-01-14, Simple string.,
"3 is three", '3 is three'] 
\end{verbatim}

Other data structure is called mapping, which is also known as directory or associative array:
\begin{verbatim}
# Example of a mapping
pi: 3.14
date: 2014-01-14   
name: Mr. Simple String
\end{verbatim}

Again one may use also JSON syntax with mapping enclosed in braces:
\begin{verbatim}
# Compact mapping
{ pi: 3.14, date: 2014-01-14,   
name: Mr. Simple String }
\end{verbatim}

Mappings and lists may by mutually nested:
\begin{verbatim}
list:
    - one
    - two
    - 
        - three one 
        - three two
map:
    a: one 
    b: two
\end{verbatim}

A string may be split to more lines using {\it greater then} (\verb'>') and multi-line strings may be entered after {\it vertical line} (\verb'|'):
\begin{verbatim}
# single long string
one_line: >
    Single line string
    broken into two lines.
# multi line string
multi_line: |
    First line.
    Second line.
\end{verbatim}
In the first case the line breaks are replaced by space, in the second case the line breaks are preserved.
In both cases the leading indentation is removed.


\subsubsection{Tags}
YAML format defines a~syntax for explicit specification of types of values including the types specific to an application.
The application specific tags are used by Flow123d to specify particular implementation of various algorithms or data types.
The general syntax of tags is quite complicated, so we present only the syntax used in the Flow123d input.
\begin{verbatim}
    field_a: !FieldFormula
        value: !!str "5 * x" 
    field_b:  !FieldFormula "5 * x"   
\end{verbatim}

The \verb'field_a' have specified evaluation algorithm \verb'FieldFormula', the key value have explicitly specified the default tag \verb'str'.
Note that default types are detected automatically and need not to be specified. On the third line we use even more compact way to 
express the same thing. Further details about usage of tags in Flow123d follows in Section \ref{sec:abstract}.

\subsubsection{References}
Anchors and references allows to reduce redundancy in the input data:
\begin{verbatim}
aux_name: &anchor_name John Smith
aux_man: &common_man
    sex: man
    city: Prague
    
people:
   - << *_common_man
     name: John Paul
   - << *_common_man
     name: *anchor_name
\end{verbatim}
On the first line, we define the anchor \verb'&anchor_name' for the value \verb'John Smith'. On the second line, 
we define the anchor \verb'&common_man' for the dictionary. Later, we use \verb'<<' to inject the dictionary 
referenced by \verb'*common_man'. Finally the anchor \verb'&anchor_name' is referenced by \verb'*anchor_name' to reuse the 
name \verb'John Smith'.

Ignoring the auxiliary keys \verb'aux_name' and \verb'aux_man' this is equivalent to:
\begin{verbatim}
people:
   - sex: man
     city: Prague
     name: John Paul
   - sex: man
     city: Prague
     name: John Smith
\end{verbatim}


\subsubsection{Gotchas}
\begin{itemize}
 \item Unquoted string values can not contain characters: colon \verb':', hash \verb'#', 
 brackets \verb'[]', braces \verb'{}', less then \verb'<', vertical bar \verb'|'.
 \item For indentation one can use only spaces; tabs are not allowed. However, your editor may automatically convert tabs to spaces.
 \item Boolean special strings must be quoted if you want to express a~string. This is not problem for the Flow123d input.
 \item Numbers starting with digit zero are interpreted as octal numbers. 
\end{itemize}

\subsection{Flow123d input types}
\label{sec:input_types}
Flow123d have a~subsystem for definition of the structure of the input file including preliminary checks for the 
correctness of the values. This subsystem works with elementary data types:
\begin{itemize}
 \item {\it Bool} corresponds to the YAML Boolean values
 \item {\it Double}, {\it Integer} initialized from YAML numeric values. 
 \item {\it String}, {\it FileName}, {\it Selections} initialized from YAML strings.
\end{itemize}
Numerical values have defined valid ranges. FileName values are used for reference to external files either for input or for output.
Selection type defines a~finite number of valid string values which may appear on the input. 
These elementary types are further organized into Records and Arrays in order to specify strongly typed definition of the 
input data file. Array and Records forms so called input structure tree (IST).

In order to make {\it "simple things simple and complex things possible" (Alan Kay)} the input system provides
so called {\it automatic conversions}. If the YAML type on input does not match the expected data type automatic conversion tries to convert 
the input to the expected type. Automatic conversion rules for individual composed types follows.

\subsubsection{Record (YAML Mapping, JSON object)}
A Record is initialized from the YAML mapping. However, in contrast to YAML mappings 
the Records have fixed keys with fixed types. 
This is natural as Records are used for initialization of C++ objects which 
are also strongly typed. Every Record type have unique name and have defined list of its keys.
Keys are lowercase strings without spaces, possibly using digits and underscore. Every key has
a type and default value specification. Default value specification can be:
\begin{description} 
 \item[obligatory] --- means no default value, which has to be specified at input. 
 \item[optional] --- means no default value, but value is needs not to be specified. Unspecified value usually means that you turn off some functionality.
 \item[default at declaration] --- the default value is explicitly given in declaration and is automatically provided by the input subsystem if needed
 \item[default at read time] --- the default value is provided at read time, usually from some other variable. In the documentation, 
 there is only textual description where the default value comes from.
\end{description}

Records that have all keys with default value or optional safe the single key $K$ may support autoconversion from an input of the type that match 
the type of the key $K$. For example:
\begin{verbatim}
  mesh: "mesh_file.msh"
\end{verbatim}
is converted to:
\begin{verbatim}
  mesh:
    mesh_file: "mesh_file.msh"
    regions: null
    partitioning: any_neighboring
    print_regions: false
    intersection_search: BIHsearch
\end{verbatim}
with the key \verb'regions' being optional and the last three keys having its default values. 


\subsubsection{Array (YAML List, JSON array)}
An Array is initialized from a~YAML list. But, in opposition to the YAML mapping, the values in a~single Array 
have all the same type. So the particular Array type is given by the type of its elements and a~specification of its size range.

Automatic conversion performs kind of transposition of the data. It simplifies input of the list of records (or arrays) 
with redundant structure, e.g. consider input
\begin{verbatim}
  list:
    a: [1,2)
    b: 4
    c: [5,6]
\end{verbatim}
Assuming that key \verb'list' have the type Array of Records and keys \verb'a', \verb'b', \verb'c' are all numerical scalars this input is equivalent to
\begin{verbatim}
  list:
    - a: 1
      b: 4
      c: 5
    - a: 2
      b: 4
      c: 6
\end{verbatim}
The rule works as follows, if a~key $K$ should have type Array, but some other type is on the input, 
we search through the input under the key $K$ for all Arrays $S$ standing instead of scalars.
All these arrays must have the same length $n$. Then the $i$-th element of the key $A$ array is
copy of the input keeping only $i$-th elements of the Arrays $S$.
As a~special case if there are no Arrays $S$ a~list with single element equal to the input is created.
Only this simplest conversion to an Array is applied if inappropriate type is found 
while the transposition is processed.





\subsubsection{Abstract}
\label{sec:abstract}
An Abstract type allows a~certain kind of polymorphism corresponding to a~pure abstract class in C++ or to an interface in Java. 
Every Abstract type have unique name and set of Records that implements the Abstract. The particular type must be provided on input through the YAML tag.
See description of \hyperlink{sec:Fields}{Fields} below for examples.

An Abstract type may have specified the default implementation. If this default Record supports automatic conversion from one of its keys
we can see it as an automatic conversion from that key to the Abstract. For example
\begin{verbatim}
 conductivity: 2.0
\end{verbatim}
where conductivity is of Abstract type \verb'Field' with scalar values, is in fact converted to
\begin{verbatim}
 conductivity: !FieldConstant
    value: 2.0
\end{verbatim}
as the \verb'FieldConstant' is default implementation of the field and it is auto=convertible from the key \verb'value'. 

\subsubsection{Flow123d specific tags}
\label{sec:spec_tags}
Currently just two specific tags are implemented, both allowing inclusion of data in other files.

{\bf Include other YAML file} The tag \verb'!include' can be used to read a~key value from other YAML file.
Path to the file is specified as the value of the key. A relative path is rooted in the folder of the main input file.
A particular type of an Abstract key is specified as a~composed tag \verb'!include,<TYPE>'.

Example, the main input file:
\begin{verbatim}
    flow123d_version: 2.0.0
    problem: !Coupling_Sequential
        description: Test8 - Steady flow with sources
        mesh:
            mesh_file: ../00_mesh/square_1x1_shift.msh
        flow_equation: !include,Flow_Darcy_MH
            input_darcy.yaml
\end{verbatim}
Content of \verb'input_darcy.yaml', included Record:
\begin{verbatim}
    nonlinear_solver:
        linear_solver: !Petsc
    input_fields:
        darcy_input_fields.yaml
    balance: {}
    output_stream: 
        file: ./flow.pvd
\end{verbatim}
Content of \verb'darcy_input_fields.yaml', included Array:
\begin{verbatim}
    - region: plane
        anisotropy: 1
        water_source_density: !FieldFormula
        value: 2*(1-x^2)+2*(1-y^2)
    - region: .plane_boundary
        bc_type: dirichlet
        bc_pressure: 0    
\end{verbatim}

{\bf Include general CSV data}
The custom tag \verb'include_csv' can be used to include a~table (e.g. coma separated values, CSV file) as an Array of Records. 
Every line of the input table is converted to a~single element of the Array.
The tag is followed by a~Record with several keys to specify format of the data:
\begin{description}
  \item {\bf\verb'file'} \\A valid path to a~text data file. Relative to the main input file.
  \item {\bf\verb'separator'}\\ A string of characters used as separators of the values on the single line (default is coma ",").
  Tab and space are always added. Double quotas can be used to express string values containing separators, 
  backslash can be used to escaping any character with special meaning. Consecutive row of separators is interpreted as a~single separator. 
  \item {\bf\verb'n_head_lines'}\\ Skip given number of lines at the beginning.
  \item {\bf\verb'format'}\\ An input structure of a~single element in the resulting array. Type of Abstracts must be same through 
  the whole resulting Array. String scalar values with a~placeholder \verb|'$<column>'| will be replaced by the value 
  at corresponding column in the input file.
\end{description}

Current implementation have substantial limitation as it can not be combined with auto conversions. This makes these includes
little bit verbose. For example consider this section from a~main input file:
\begin{verbatim}
    ...
    substances: [A, B]
    ...    
    input_fields:
    - region: A
      porosity: !FieldTimeFunction
        time_function: !include_csv
          values:
            file: data.txt  
            separator: " "
            n_head_lines: 1
            format: 
              time: #0
              value: #1
              
    - region: .boundary_A        
      bc_conc: 
        - !FieldTimeFunction    # Substance A
          time_function: !include_csv
            values:
              file: data.txt  
              separator: " "
              n_head_lines: 1
              format: 
                time: #0
                value: #2
        - !FieldTimeFunction    # Substance B
          time_function: !include_csv
            values:
              file: data.txt  
              separator: " "
              n_head_lines: 1
              format: 
                time: #0
                value: #3
\end{verbatim}
Content of \verb'data.txt':
\begin{verbatim}
time    porosity        bc_conc_X     bc_conc_Y
0.0     0.01            1.0           0.6
0.1     0.015           0.9           0.5
0.2     0.03            0.8           0.4
\end{verbatim}

This together will be equivalent to:
\begin{verbatim}
    input_fields:
    - region: A
      porosity: !FieldTimeFunction
        time_function: 
          - time: 0.0
            value: 0.01
          - time: 0.1
            value: 0.015
          - time: 0.2
            value: 0.03
           
    - region: .boundary_A        
      bc_conc: !FieldTimeFunction
        time_function: 
          - time: 0.0
            value: [ 1.0, 0.6]
          - time: 0.1
            value: [ 0.9, 0.5]
          - time:  0.2
            value: [ 0.8, 0.4]
\end{verbatim}

So in this particular case it would be simpler to write data directly into the main file. The include from 
the text table is efficient for the long time series.




\subsection{Input subsystem}
This section provides some implementation details about the Flow123d input subsystem. This may be helpful to better understand behavior of the program for 
some special input constructions.

\begin{figure}[!hb]
 \begin{center}
 \includegraphics[scale=0.4]{\fig/input_subsystem.pdf}
 % input_subsystem.pdf: 0x0 pixel, -2147483648dpi, 0.00x0.00 cm, bb=
 \caption{Structure of the input subsystem. HDF5 format not yet supported.}
 \label{fig:input_subsystem}
 \end{center}
\end{figure}

The input subsystem of Flow123d is designed with the aim to provide uniform initialization of 
C++ classes and data structures. The scheme of the input is depicted on Figure \ref{fig:input_subsystem}.
The structure of the input file is described by the Input Structure Tree (IST) consisting of the input objects describing 
the types discussed in the previous Section \ref{sec:input_types}. The structure of the tree mostly follows follows the structure of the computational classes.

When reading the input file, the file is first parsed by the specific format parser. Using a~common interface to the parsed data, the 
structure of the data is checked against the IST and the data are pushed into the storage tree. If the input data and IST do not match
the automatic conversions described above are applied, where appropriate.
An accessor object to the root data record is the result of the file reading. The data can be retrieved through the 
accessors which combine raw data of the storage with their meaning described in IST. The IST can be written out in the JSON format
providing the description of the input file structure. This IST file is used both for generation of the input reference in HTML and \LaTeX
formats and for the Model editor --- specialized editor for the input file that is part of the GeoMop tools currently in development.

While the recommended format of the input file is YAML the JSON format can be used as well. This may be useful in particular if the input file
should be machine generated. Although the JSON format is technically subset of the YAML format. We use separate parser and use special keys in order to
mimic tags and references supported by the YAML. The type of an abstract is specified by the key \verb'TYPE'. A reference is given by a~record with the only key 
\verb'REF' which contains a~string specifying the address of the value that should be substituted.



\section{Important Record Types of Flow123d Input}
Complete description of the input structure tree can be generated into HTML or LaTeX format. While the former one provides better interactivity 
through the hyperlinks the later one is part of this user manual. The generated documentation provides whole details for all keys, but 
it may be difficult to understand the concept of the input structures. This section is aimed to provide this higher level picture.

\subsection{Mesh Record}
\label{sec:Mesh}
The \hyperA{IT::Mesh}{mesh record} provides initialization of the computational mesh consisting of points, lines, triangles and tetrahedrons in the 3D ambient space.
Currently, we support only GMSH mesh file format \href{http://geuz.org/gmsh/doc/texinfo/gmsh.html#MSH-ASCII-file-format}{MSH ASCII}. 
The input file is provided by the key \hyperA{Mesh::mesh-file}{{\tt mesh\_file}}. The file format allows to group elements into {\it regions} 
identified by a~unique label (or by ID number). The regions with labels starting with the dot character are treated as {\it boundary regions}. 
Their elements are removed from the computational domain, however they can be used to specify boundary
conditions. Other regions are called {\it bulk regions}. Every element lies directly in just one {\it simple region} while the simple regions may be 
grouped into composed regions called also region sets. A simple region may be part of any number of composed regions.
Initial assignment of the simple regions to the elements is given by the physical groups of the input GMSH file. Further
modification of this assignment as well as creation of new simple or composed regions can be done 
through the list of operations under the key \hyperA{Mesh::regions}{{\tt regions}}. The operations are performed in the order given by the input.
Operation \hyperA{IT::From-Id}{{\tt From\_Id}} sets the name of a~simple region having only ID in the input GMSH file. Operation 
\hyperA{IT::From-Label}{{\tt From\_Label}} can rename a~simple region. Operation \hyperA{IT::From-Elements}{{\tt From\_Elements}}
assign new simple region to the given list of elements overwriting their region given by the input mesh file. Finally operations 
\hyperA{IT::Union}{{\tt Union}}, \hyperA{IT::Difference}{{\tt Difference}} and \hyperA{IT::Intersection}{{\tt Intersection}}
implements standard set operations with both simple and complex regions resulting in new composed regions.




\subsection{Input Fields}
Input of every equation contains the key \verb'input_fields' used consistently for the input of the equation parameters 
in form of general time--space dependent fields.  The input fields are organized into a~list of {\it field descriptors}, see 
e.g. \hyperA{IT::Flow-Darcy-MH-Data}{\tt Data} record, the field descriptor of the Darcy flow equation.
The field descriptor is a~Record with keys 
\hyperA{Flow-Darcy-MH-Data::time}{{\tt time}}, 
\hyperA{Flow-Darcy-MH-Data::region}{{\tt region}}, 
\hyperA{Flow-Darcy-MH-Data::rid}{{\tt rid}}
and further keys corresponding to the 
names of input fields supported by the equation. The field descriptor is used to prescribe
a change of one or more fields in particular time (key \verb'time') and on particular region given  by the name (key \verb'region', preferred way) 
or by the region id (key \verb'rid'). 
The array is processed sequentially and latter values overwrite the previous ones. Change times of a~single field must form a~non-decreasing sequence.
Changes in fields given by the fields descriptor are interpreted as discontinuous changes of the changed fields
and equations try to adopt its time stepping to match these time points. This is in contrast with changes of the field values given by
the evaluation algorithms, these are always assumed to be continuous and the time steps are not adapted. 



Example:
\begin{verbatim}
input_fields:
  - # time=0.0  - default value
    region: BULK
    conductivity: 1   # setting the conductivity field on all regions
  - region: 2d_part
    conductivity: 2  # overwriting the previous value
  - time: 1.0
    region: 2d_part
    conductivity: !FieldFormula
      # from time=1.0 we switch to the linear function in time
      value: 2+t
  - time: 2.0
    region: 2d_part
    conductivity: !FieldElementwise
      # from time=2.0 we switch to elementwise field, but only
      # on the region "2d_part"
      gmsh_file: ./input/data.msh
      field_name: conductivity
\end{verbatim}

% TODO: describe automatic conversions in a separate paragraph havin an overview of these structural exceptions
% and can refer to the section. 

\subsubsection{Field Algorithms}
\label{sec:Fields}\hypertarget{sec:Fields}{}

A general time and space dependent, scalar, vector, or  tensor valued function can be specified through the family of abstract records 
\verb'Field:R3 -> X', where $X$ is a~type of value returned by the field, which can be:
\begin{itemize}
 \item $T$ --- scalar valued field, with scalars of type $T$
 \item $T[d]$ --- vector valued field, with vector of fixed size $d$ and elements of type $T$
 \item $T[d, d]$ --- tensor valued field, with square tensor of fixed size and elements of type $T$
\end{itemize}
the scalar type $T$ can be one of
\begin{itemize}
 \item {\bf Real} --- scalar real valued field
 \item {\bf Int}  --- scalar integer valued field
 \item {\bf Enum} --- scalar non negative integer valued field. Values on the input are of the type Selection.
\end{itemize}

Each of these abstract records have the same set of descendants which implement various evaluation algorithms of the fields. These are
\begin{description}
 \item[FieldConstant] --- field that is constant both in space and time
 \item[FieldTimeFunction] --- field that is constant in space and continuous in time. Values are interpolated (currently only linear interpolation) from 
 the sequence of time-value pairs provided on input.
 %
 \item[FieldFormula] --- field that is given by a runtime parsed formula. % with a simplified Python and Numpy syntax. 
Since version 3.9.0 we use our own library \href{https://github.com/flow123d/bparser}{BParser} however 
we preserve the formular syntax according to the \href{http://warp.povusers.org/FunctionParser/fparser.html}{FParser}.
Formulas are converted from the FParser syntax to that of BParser for the sake of backward compatibility,
however, we can not rule out minor incompatibilities. The BParser will be used exclusively from the version
4.0.0 with its Python and Numpy syntax, support for vector and tensor valued expressions and use SIMD
operations to achieve nearly peak CPU performance. 
 
The formula can contain following literals: constants {\tt E} and {\tt Pi}, names of other fields of the same equation, 
and coordinates: {\tt t}, {\tt x}, {\tt y}, {\tt z}, {\tt d}. First is the 
simulation time, the coordinate $d$ is a special field that evaluates to the depth from the surface. This is 
defined as a distance to the intersection of the vertical (Z axis) line with the outer boundary that has smallest 
Z coordinate greater then the evaluation point. Any other field of the same equation can be used 
in the formula as long as the cyclic dependencies are avoided. 

 \item[FieldPython] --- out of order in version 3.9.0, will be improved for the version 4.0.0.
 \item[FieldFE] --- discretized fully heterogeneous field \hyperA{FieldFE::field-name}{{\tt field\_name}} read from a file in GMSH or VTK format with path given by 
 the key \hyperA{FieldFE::mesh-data-file}{{\tt mesh\_data\_file}}. Two regimes selected by the key 
 \hyperA{FieldFE::input-discretization}{{\tt input\_discretization}} are currently supported: 
 the default value \hyperA{FE-discretization::element-data}{{\tt element\_data}} 
 for the elementwise constant (P0) discretization and 
 \hyperA{FE-discretization::native-data}{{\tt native\_data}}
 for the internal discretization. 
 
 The element data are provided by the file in GMSH or VTK format. The key 
 \hyperA{FieldFE::interpolation}{{\tt interpolation}}
 indicates interpretetion of the
 input data. The value \hyperA{interpolation::identic-mesh}{{\tt identic\_mesh}}
 can be used only with the GMSH format if the values are given for the 
 same element IDs as used in the file of the common computational mesh. It is sufficient to have 
 data file with only the \verb|$ElementData| 
 sections, see Section \ref{mesh_file} for the complete format overview. The default value 
 \hyperA{interpolation::equivalent-mesh}{{\tt equivalent\_mesh}} assumes the data mesh same as the computational mesh, 
 but possibly with different 
 numbering of the elements. The mapping between meshes is determined by the node coordinates, so 
 the complete mesh information must be provided with the data. Finaly there are two options for the case of different data mesh. 
 For the case \hyperA{interpolation::P0-gauss}{{\tt P0\_gauss}}, the value on an element of the computational mesh is given as an integral average approximated by the Gauss quadrature of 
 the order 4. The values at the quadrature points are determined by the values on the data mesh. 
 The case \hyperA{interpolation::P0-intersection}{{\tt P0\_intersection}} works only for the boundary fields. The boundary of the computational mesh is intersected
 with the data mesh, a single boundary element is decomposed into its intersection $S_i$ with the elements of the data mesh, and 
 the field value $f$ is determined as a weighted average of the data elements values $v_i$:
 \[
    v = \frac{\sum_i |S_i| v_i}{\sum_i |S_i|}
 \]

 The case of native data, is only available for the VTK file produced by the Flow123d. It can be used to pass data between 
calls of the Flow123d on the same mesh using the original discretization. This is particularly usefull for the mixed-hybrid 
 finite elements used by the Darcy flow model. 
 
 The key \hyperA{FieldFE::default-value}{{\tt default\_value}} provides to the computational elements not found in the data file.
The input data formats can contain field values for a sequence of discrete times, the key 
\hyperA{FieldFE::read-time-shift}{{\tt read\_time\_shift}} together with the 
\hyperA{FieldFE::time-unit}{{\tt time\_unit}} can be used to add given period to the data times in order to match time 
frame of the simulation.  
 \end{description}

\subsubsection{Field automatic conversions} 
Several automatic conversions are implemented to simplify field specifications. 
Scalar values can be used to set constant vectors or tensors. 
Vector value of size $d$ can be used to set diagonal tensor $d\times d$.
Vector value of size $d(d-1)/2$, e.g. $6$ for $d=3$, can be used to set symmetric tensor. 
These rules apply only for \verb|FieldConstant|. Still supported, but deprecated for \verb|FieldFormula|.
Moreover, all Field abstract types have default value \verb'TYPE=FieldConstant'. 
Thus you can just use the constant value instead of the whole record.

Examples:
\begin{verbatim}
input fields:
   - conductivity: 1.0
     # is equivalent to
   - conductivity: !FieldConstant
        value=1.0
   
   - anisotropy: [1 ,2, 3] # diagonal tensor
     # is equivalent to
   - anisotropy: !FieldConstant
        value=[[1,0,0],[0,2,0],[0,0,3]]

     # concentration for 2 components   
   - conc: !FieldFormula  ["x+y", "x+z"]
     # is equivalent to
   - conc: 
       - !FieldFormula
         value: "x+y"
       - !FieldFormula
         value: "x+z"
\end{verbatim}

\subsubsection{Field Units}
Every field (e.g. conductivity or storativity) have specified unit in terms of powers of the base SI units. 
The user, however, may set input in different units specified by the key \verb'unit' 
supported by every field algorithm. The key have type \hyperA{IT::Unit}{\tt Unit} record, auto convertible from its only key 
\verb'unit' of the string type. Effectively the \hyperA{IT::Unit}{\tt Unit} is a string with particular syntax. 
The unit formula is evaluated  into a~coefficient and an SI unit. The resulting SI unit 
must match expected SI unit of the field, while the input value 
of the field (including values from external files or returned by Python functions)  
is multiplied by the coefficient before further processing.

The syntax of unit formula is: {\tt <UnitExpr>;<Variable>=<Number>*<UnitExpr>;...,}
where {\tt <Variable>} is a~variable name and {\tt <UnitExpr>} is a~units expression
which consists of products and divisions of terms, where a~term has form \verb'<Base>^<N>', 
where {\tt <N>} is an integer exponent and {\tt <Base>} is either a~base SI unit, 
a derived unit, or a~variable defined in the same unit formula.
Example, unit for the pressure head: 
\begin{verbatim}
   pressure_head: !FieldConstant      # [m] expected
     value: 100                       # [MPa] 
     unit:  MPa/rho/g_; rho = 990*kg*m^-3; g_ = 9.8*m*s^-2     
\end{verbatim}

Standard single letter prefixes: a,f,p,n,u,m,d,c,h,k,M,G,T,P,E\\
are supported for the basic SI units: m,g,s,A,K,cd,mol\\
and for the derived SI units: N, J, W, Pa, C, D, l.

Moreover several specific units are supported: 
t = 1000 kg
min = 60 s 
h = 60 min
d = 24 h
y = 365.2425 d




\subsection{Output Records}
Output from the models is controlled by an interplay of following records: \hyperA{IT::OutputStream}{{\tt OutputStream}},
\hyperA{IT::Balance}{{\tt Balance}}, and \hyperA{IT::EquationOutput}{{\tt EquationOutput}}. The first two are part of the 
records of so called {\it balance equations} which provides complete description of some balanced quantity. 
Every such equation have its own balance output controlled by the \verb'Balance' record and its own output stream for the 
spatial data controlled by the \verb'OutputStream' record. Further every equation with its own output fields 
(every input field is also output field) have the \verb'EquationOutput' record to setup output of its fields.

\subsubsection{Balance}
The balance output is performed in times given by the key \hyperA{Balance::times}{{\tt times}}
with type \verb'TimeGrid' described \hyperlink{sec:TimeGrid}{below}. Setting the key \hyperA{Balance::add-output-times}{{\tt add\_output\_times}} 
to \verb'true' the set of balance output times is enriched by the output times of the output stream of the same equation.

\subsubsection{OutputStream}
Set the file format of the output stream, possibly setting the output name, however the default value for the file name is preferred and the corresponding key 
\hyperA{OutputStream::file}{{\tt file}} is obsolete.
The time set provided by the optional key \hyperA{OutputStream::times}{{\tt times}} is used as a~default time set for a~similar key in associated 
\verb'EquationOutput' records. Finally, the key \hyperA{OutputStream::observe-points}{{\tt observe\_points}} is used to specify observation points
in which the associated equation output evaluates the observed fields.

\subsubsection{EquationOuput}
The output of the fields can be done in two ways: full spatial information saved only at selected time points in form of 
VTU or GMSH file, or full temporal information saved for every computational time, but only in selected output points.
The list of fields for spatial output is given by key \hyperA{EquationOutput::fields}{{\tt fields}} while the fields 
evaluated in the observation points are selected by the key \hyperA{EquationOutput::observe-fields}{{\tt observe\_fields}}.
The outputs times for the spatial output can be selected individually for every field in the 
\hyperA{EquationOutput::fields}{{\tt fields}} however the default list of output times is given by the key
\hyperA{EquationOutput::times}{{\tt times}} which can by optionally extended by the list of input times
using the key \hyperA{EquationOutput::add-input-times}{{\tt add\_input\_times}}.

\subsubsection{TimeGrid Array}
\hypertarget{sec:TimeGrid}{}

An array of the \hyperA{IT::TimeGrid}{{\tt TimeGrid}} records may be used to setup a~sequence of times. Such sequence is used in particular 
to trigger various types of output. A single TimeGrid represents a~regular grid of times with given start time, end time and step.
