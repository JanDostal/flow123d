\makeatletter
\def\maxwidth{\ifdim\Gin@nat@width>\linewidth\linewidth\else\Gin@nat@width\fi}
\def\maxheight{\ifdim\Gin@nat@height>\textheight\textheight\else\Gin@nat@height\fi}
\makeatother
% Scale images if necessary, so that they will not overflow the page
% margins by default, and it is still possible to overwrite the defaults
% using explicit options in \includegraphics[width, height, ...]{}
\setkeys{Gin}{width=\maxwidth,height=\maxheight,keepaspectratio}
% Set default figure placement to htbp
\makeatletter
\def\fps@figure{htbp}
\makeatother

\section{Fractures and dual porosity}

File: \texttt{06\_frac\_dualpor.yaml}

\subsection{Description}

This is a variant of \texttt{04\_frac\_diffusion.yaml}. Instead of
diffusion we consider advective transport with dual porosity.

\subsection{Input}

Dual porosity substitutes dead-end fractures in this task. The
dual-porosity parameter \texttt{diffusion\_rate\_immobile} was
calibrated to the value 5.64742e-06 for identical results with the model
with the dead-end fractures. Other settings of transport are identical
to the diffusion model.

The dual porosity model is set by the following lines:

\begin{verbatim}
reaction_term: !DualPorosity
  input_fields:
    - region: rock
      init_conc_immobile: 0
    - region: flow_fractures
      diffusion_rate_immobile: 5.64742e-06
      porosity_immobile: 0.01
      init_conc_immobile: 0
    - region: deadend_fractures
      init_conc_immobile: 0
\end{verbatim}

\subsection{Results and comparison}

Results of calibration of the model with dual porosity and model with
flow in dead-end fractures (file \texttt{06\_frac\_nodualpor.yaml}) is
depicted in Figure \ref{fig:calib}.

\begin{figure}
\hypertarget{fig:calib}{%
\centering
\includegraphics{tutor_figures/06_mass_flux.pdf}
\caption{Results of calibration.}\label{fig:calib}
}
\end{figure}
