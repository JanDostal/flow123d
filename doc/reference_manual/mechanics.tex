\section{Mechanics}
\def\ee{\vc\varepsilon}
\newcommand{\eq}[1]{\begin{equation}#1\end{equation}}
\def\nn{\vc n}
% \def{\tr}{\operatorname{tr}}
\def\uu{\vc u}

Deformation of the porous media is modelled by the stationary linear elasticity equation:
\eq{\label{eq:lin_el} -\div(\delta\tn \sigma(\uu)) = \delta\vc f + \vc f_C + \vc f_H. }
Here $\uu$ \units{}{1}{} is the displacement vector field with 3 components, the stress tensor is given by the Hooke law
\eq{ \tn\sigma(\uu) = \tn C\ee(\uu) = 2\mu\ee(\uu) + \lambda(\tn I:\ee(\uu))\tn I, }
and the Lam\'e parameters are determined in terms of the \hyperA{Mechanics-LinearElasticity-FE-Data::young-modulus}{Young modulus} $E$ [$\mathrm{Pa}$] and \hyperA{Mechanics-LinearElasticity-FE-Data::poisson-ratio}{Poisson's ratio} $\nu$ \units{}{}{}:
\eq{ \mu = \frac{E}{2(1+\nu)},\quad \lambda = \frac{E\nu}{(1+\nu)(1-2\nu)}. }
The strain tensor in $\Omega_d$ is defined as follows:
\eq{ \ee(\uu) = \frac12(\nabla\uu_d+\nabla\uu_d^\top) + \begin{cases}0&\mbox{ if }d=3,\\\frac1\delta\sum_{i=1}^n\uu_{d+1}^i\otimes_s\nn_{d+1}^i & \mbox{ else}.\end{cases} }
Here $\vc a\otimes_s\vc b:=\frac12(\vc a\otimes\vc b+\vc b\otimes\vc a)$.

The symbol $\vc f$ stands for the \hyperA{Mechanics-LinearElasticity-FE-Data::load}{body load} [$\mathrm{Nm}^{-3}$].


\paragraph{Hydromechanical coupling.}
The mechanics equation \eqref{eq:lin_el} is coupled to flow by the term
\eq{ \vc f_H = -\nabla(\delta\alpha p), \quad p = \varrho_l g h, }
where $p$ [$\mathrm{Pa}$] is the pressure, $\alpha$ \units{}{}{} is the \hyperA{Coupling-Iterative-Data::biot-alpha}{Biot coefficient}, $\varrho_l$ \units{1}{-3}{} is the \hyperA{Coupling-Iterative-Data::fluid-density}{fluid density} and $g$ \units{}{1}{-2} is the \hyperA{Coupling-Iterative-Data::gravity}{gravitational acceleration}.
Conversely, the deformation affects the flow via the additional term
\eq{\label{eq:fluid_source_div_u} F_M = -\partial_t(\delta\alpha\widetilde\div\uu) }
on the right hand side of \eqref{eq:continuity}.
The expression $\widetilde\div\uu$ is defined as follows:
\eq{ \widetilde\div\uu_d = \div\uu_d + \begin{cases}\frac{\delta_{d+1}}{\delta_d}\sum_{i=1}^n\uu^i_{d+1}\cdot\nn^i_{d+1} & \mbox{ if }d\in\{1,2\},\\0 & \mbox{ else. }\end{cases} }
The numerical solution of coupled hydro-mechanical problems is solved by an iterative splitting, where in order to achieve convergence the flow equation is modified as follows:
\eq{ \partial_t(\delta(S+S_{extra})h) + \div\vc q = F + F_M + \partial_t(\delta S_{extra} h_{old}). }
Here $h_{old}$ is the previous value of piezometric head in the iteration process and $S_{extra}$ is an extra storativity coefficient whose value affects the rate of convergence.
It can be manually tuned using the \hyperA{Coupling-Iterative::iteration-parameter}{iteration parameter}.


\paragraph{Boundary conditions.}
Given the decomposition $\partial\Omega_d=\Gamma_D\cup\Gamma_{DN}\cup\Gamma_T\cup\Gamma_S$, we prescribe the following \hyperA{Mechanics-LinearElasticity-FE-Data::bc-type}{boundary conditions}:
\begin{itemize}
\item \textbf{displacement} condition prescribes
\eq{ \uu = \uu_D \mbox{ on }\Gamma_D }
via \hyperA{Mechanics-LinearElasticity-FE-Data::bc-displacement}{given displacement} $\uu_D$ \units{}{1}{}.
\item \textbf{displacement\_n}: Displacement is prescribed only in the normal component, in tangent directions(s) zero traction is assumed:
\eq{ \left.\begin{aligned}\uu\cdot\nn &= \uu_D\cdot\nn\\
(\sigma(\uu)\nn)_\tau &= \vc 0\end{aligned}\right\} \mbox{ on }\Gamma_{DN}. }
Here $\vc a_\tau := \vc a - (\vc a\cdot\nn)\nn$ is the projection of a vector $\vc a$ to the tangent plane of the boundary
\item \textbf{traction} condition (default) is imposed via \hyperA{Mechanics-LinearElasticity-FE-Data::bc-traction}{given traction} $\vc t_N$ [$\mathrm{Pa}$]:
\eq{ \sigma(\uu)\nn = \vc t_N \mbox{ on }\Gamma_T. }
\item \textbf{stress} condition is the same type as \textbf{traction}, but instead of traction the user supplies the full \hyperA{Mechanics-LinearElasticity-FE-Data::bc-stress}{stress tensor} $\tn\sigma_N$ [$\mathrm{Pa}$]:
\eq{ \sigma(\uu)\nn = \tn\sigma_N\nn \mbox{ on }\Gamma_S. }
\end{itemize}


\paragraph{Communication between dimensions.}
The mechanical interaction on the interface between $\Omega_{d+1}$ and $\Omega_d$ is realized via the traction condition on the boundary of $\Omega_{d+1}$:
\eq{ \delta_{d+1}(\sigma(\uu^i_{d+1})-\alpha_{d+1}p_{d+1}\tn I)\nn^i_{d+1} = \vc t^i_{d+1,d}, }
where
\eq{ \vc t^i_{d+1,d} = \sigma^U_{d+1,d}\delta_{d+1}\left(\frac{2\delta_{d+1}}{\delta_d}\tn C_d\left((\uu_{d+1}^i-\uu_d)\otimes\nn_{d+1}^i\right)-\alpha_d p_d\tn I\right)\nn_{d+1}^i }
and $\sigma^U_{d+1,d}$ \units{}{}{} is the \hyperA{Mechanics-LinearElasticity-FE-Data::fracture-sigma}{transition coefficient}.
The force term in $\Omega_d$ due to the interaction with $\Omega_{d+1}$ is
\eq{ \vc f_{Cd} = \begin{cases}\sum_{i=1}^n\vc t^i_{d+1,d} & \mbox{ if }d\in\{1,2\},\\\vc 0 & \mbox{ else}.\end{cases} }




