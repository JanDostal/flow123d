% Copyright (C) 2007 Technical University of Liberec.  All rights reserved.
%
% Please make a following reference to Flow123d on your project site if you use the program for any purpose,
% especially for academic research:
% Flow123d, Research Centre: Advanced Remedial Technologies, Technical University of Liberec, Czech Republic
%
% This program is free software; you can redistribute it and/or modify it under the terms
% of the GNU General Public License version 3 as published by the Free Software Foundation.
%
% This program is distributed in the hope that it will be useful, but WITHOUT ANY WARRANTY;
% without even the implied warranty of MERCHANTABILITY or FITNESS FOR A PARTICULAR PURPOSE.
% See the GNU General Public License for more details.
%
% You should have received a copy of the GNU General Public License along with this program; if not,
% write to the Free Software Foundation, Inc., 59 Temple Place - Suite 330, Boston, MA 021110-1307, USA.
%


\section{Introduction}
Flow123d is a~software for simulating water flow, reactionary solute transport, heat transfer, and mechanics in a~heterogeneous
porous and fractured medium. In particular, it is well-suited for simulations of underground processes in a~granite rock.
The program is able to describe explicitly processes in 3D medium, 2D fractures, and 1D channels and exchange between 
domains of different dimensions. Therefore, the computational mesh is a~collection of tetrahedra, triangles, and line segments.

Two water flow models are available: a model for a~saturated medium based on the Darcy law
and a~model for a~partially saturated medium described by Richards' equation.
Both models use the mixed-hybrid finite element method for the space discretization and the implicit Euler method for the time discretization.
Both models can also switch between a~transient case and a~sequence of steady states within a~single simulation. The model for unsaturated medium uses
a lumped variant of the mixed-hybrid method to satisfy the maximum principle and to guarantee stability for short time steps.

In the current version,  only the model for saturated media can be sequentially coupled  with the transport models, including
two models for the solute transport and one model for the heat transfer.

The first solute transport model can deal only with a~pure advection of several substances without any diffusion-dispersion term. It uses
the explicit Euler method for the time discretization and the finite volume method for the space discretization.
The second solute transport model describes a~general advection with hydrodynamic dispersion for several substances.
It uses the implicit Euler method for time discretization and the discontinuous Galerkin method of
the first, second, or third order for the discretization in space.
The operator splitting method can be used to couple any of these two solute transport models with  
 various processes described by the reaction term. The reaction term can treat any meaningful combination of the dual porosity,
equilibrium sorptions, decays, and linear reactions.

The heat transfer model assumes equilibrium between the temperature of the rock and of the fluid phase. It uses the same numerical scheme as the second transport model, i.e., the DG method with the implicit time discretization.

The mechanical model computes linear elastic problems, taking into account the reduced dimension approach.
It can be coupled with the flow model and solve linear poroelastic problems. This coupling is realized by an iterative scheme. Nonlinear dependency between strain and hydraulic conductivity and nonlinearity due to contacts and shear stress effects is the focus of ongoing development.

The program supports the output of all input and output fields into two file formats. One can use the GMSH mesh generator and post-processor file format or the output into widely supported VTK format. In particular, we recommend Paraview software for the visualization and post-processing of the VTK data.

The program is implemented in C/C++ using the PETSc library for linear algebra. All models can run in parallel using the MPI environment; however, the scalability of the whole program is limited due to serial mesh data structures and serial outputs.


The program is distributed under GNU GPL v. 3 license and is available on the project web page:
\url{http://flow123d.github.io}

with sources on the GitHub:
\url{https://github.com/flow123d/flow123d}.


\section{Reading Documentation}
The Flow123d documentation has two main parts. Chapters \ref{chapter:getting_started} to \ref{chapter:tutorials} form a user
manual, while the last Chapter \ref{chapter:input-tree-reference} provides an input reference.
The user manual starts with Chapter \ref{chapter:getting_started} giving instructions for the installation and execution of the program.
The Chapter \ref{chapter:mathematical_models} provides a detailed description of the implemented mathematical models. The
Chapter \ref{chapter:numerical} presents used numerical methods.
The input and output file formats are documented in Chapter \ref{chapter:file-formats}.
Finally, Chapter \ref{chapter:tutorials} consists of tutorial problems.

The input reference guide, consisting only of Chapter \ref{chapter:input-tree-reference}, is automatically generated.
It mirrors the code directly and describes the whole structure of the main input file. Description
of input records, their structure, and default values are supplied there, and bidirectional links to the user manual are provided.

The document is interactive; the {\color{blue}blue text} marks the links in the document, and the {\color{magenta}magenta text} marks the web links.




\section{Installing Flow123d}
Software Flow123d requires the tool \href{https://www.docker.com}{Docker}. Docker is an open-source project that automates
the deployment of Linux applications inside software containers. Entire Flow123d software is packed in a~Docker image that
also contains necessary libraries and crucial components of the Linux operating system.

The installation process imports the Docker image into the user's machine and personalizes the Docker image.
The installation instructions for Linux and Windows operating systems are provided in the two following sections.


\subsection{Installing Flow123d on Linux}
The installation is done under a~regular user, who must be in the group 'docker'.
Download the Linux installation package archive
\begin{center}
\verb'flow123d_<version>_linux_install.tar.gz'
\end{center}
from \href{http://flow123d.github.io/}{Official pages} and extract it to any folder:
\begin{verbatim}
  $> tar -xzf flow123d_<version>_linux_install.tar.gz
\end{verbatim}
This will create a~directory \verb'flow123d_<version>'. In the next step, navigate to this directory
and execute the \verb'install.sh' script. Example output:
\begin{verbatim}
  $> cd flow123d_3.0.4
  $> ./install.sh
  Pulling docker image 'flow123d/3.0.4'
  ...
  flow123d/3.0.4
  Installation of Flow123d finished successfully.
  Run Flow123d using script fterm.sh or flow123d.sh in bin folder.
  You can start by printing the version of Flow123d:
    bin/fterm.sh run --version
\end{verbatim}
The install script will download Docker image from the \href{https://hub.docker.com/u/flow123d}{Docker Hub}.
% The script will also print additional information during personalization process.
Whole process may take several minutes (depending on user's machine performance and internet connectivity).


% \subsubsection{Alternate way to install}
% Assuming having \href{https://www.docker.com}{Docker} installed, one can simply run:
% \begin{verbatim}
%   > curl -s https://flow.nti.tul.cz/get | bash
% \end{verbatim}
% This will check user's system and download scripts \verb'flow123d' and \verb'fterm'.
% Then the user can continue by running 
% \begin{verbatim}
%   > fterm --version 3.0.4 run
% \end{verbatim}
% to open the shell of the Docker container. This will download the Docker image of the selected version
% if it is not already available.

\subsection{Installing Flow123d on Windows}
\subsubsection{Before you install}

This version uses \verb'Docker Desktop', previous versions which used \verb'Docker for Windows' and \verb'Docker Toolbox' will \textbf{stop working}.

Make sure your system fullfills following requirements in order to support \verb'Docker Desktop':
\begin{itemize}
    \item Windows 10 64bit: Pro, Enterprise or Education (1607 Anniversary Update, Build 14393 or later).
    \item Virtualization is enabled in BIOS. Typically, virtualization is enabled by default. This is different from having Hyper-V enabled. For more detail see \href{https://docs.docker.com/docker-for-windows/troubleshoot/#virtualization-must-be-enabled}{Virtualization must be enabled} in Troubleshooting.
    \item CPU SLAT-capable feature.
    \item At least 4GB of RAM.
\end{itemize}


\subsubsection{Installation}

To install Flow123d on Windows, download installer from \href{http://flow123d.github.io/}{Official pages}, execute it and follow instructions
on your screen. To make things easier, you can also \href{https://www.youtube.com/watch?v=xDR2vU-1IhM}{watch a installation video}.

If for some reason the installation failed, make sure everything below is in order:


\subsubsection{Flow123d troubleshooting}

\begin{itemize}
  \item \textbf{Powershell is installed}: On the Windows systems we require PowerShell. Windows PowerShell needs to be installed on Windows Server 2008 and Windows Vista only.
  It is already installed on Windows Server 2008 R2 and Windows 7 and higher. To install PowerShell follow instructions at
  \href{https://msdn.microsoft.com/en-us/powershell/scripting/setup/installing-windows-powershell}{Microsoft pages}.

  \item \textbf{Powershell is in the system PATH}: Make sure \verb'powershell' command is in the system PATH.
  \href{http://www.powershelladmin.com/wiki/PowerShell_Executables_File_System_Locations}{PowerShell executable location}
   is specific to the particular Windows version, but usual location is:
   \begin{verbatim}
      %SystemRoot%\system32\WindowsPowerShell\v1.0\powershell.exe
   \end{verbatim}
   To add this location to the system PATH variable follow the instructions at
   \href{https://msdn.microsoft.com/en-us/library/office/ee537574(v=office.14).aspx}{Microsoft pages}.

   \item For detailed instructions refer to \href{https://docs.docker.com/desktop/windows/}{Docker docs}.
\end{itemize}


\subsubsection{Uninstalling Flow123d}
To uninstall Flow123d, in Windows open \verb'Apps & features' (\verb'Aplikace a funkce' in Czech), find the Flow123d in the list
and click uninstall. This will only uninstall the Flow123d but not \verb'Docker Desktop'.


\subsubsection{Reinstalling Flow123d}
\label{duplicit-image}
If you are installing same version of Flow123d again, installation process will be the same,
except \verb'Docker Desktop' installation will be skipped.


\section{Running Flow123d}
\subsection{Running Flow123d on Linux}
\label{subsec:running-flow123d-on-linux}
All necessary scripts for running Flow123d are located in the \verb'bin' subdirectory of the installation directory \verb'flow123d_<version>'.
From the user's perspective, an unpleasant issue with the Docker container is that it cannot easily interact with the host file system by default.
Using the supplied scripts in \verb'bin' will make things a~lot easier for the user.
The directory \verb'bin' contains:
\begin{itemize}
	\item \verb'fterm.sh' \\
	 The script will invoke a~shell inside the Docker container and mount the user's home directory.
	 In this shell, the user has access to the system path where Flow123d is installed.
	 By default, the command \verb'flow123d' is in the \verb'PATH' variable.
	 
	\note{ On some systems, the shell's font is extremely small, you can change this behavior by right-clicking on the window bar and selecting} 
	\verb'default' \emph{or} (\verb'vychozi' in Czech) \emph{see \autoref{fig:TerminalFont}.}
	 \begin{figure}
		 \center
		 \includegraphics[width=1\textwidth]{\fig/TerminalFont.png}
		 \caption{Changing default font family and font size}
		 \label{fig:TerminalFont}
	 \end{figure}

	\item \verb'flow123d.sh' \\
	 The script will run Flow123d inside the Docker container and mount the user's home directory.
	 All arguments passed to this script will be passed to \verb'flow123d' binary file inside the container.

	\item \verb'runtest.sh' \\
	 The script will run Flow123d tests inside the Docker container and mount the user's home directory.
	 All arguments passed to this script will be passed to \verb'runtest.py' Python script inside the container.
\end{itemize}

\note{ Using the above} \verb'.sh'
\emph{scripts will mount the user's home directory to the Docker container under the same name.
Also your current working directory will be the same. The example below shows the behavior of the scripts:}
\begin{verbatim}
$> pwd
/home/username/local/flow123d_3.0.4

$> ls
bin  doc  install.sh  tests uninstall.sh

$> bin/fterm.sh
 ___ _            _ ___ ____   _
| __| |_____ __ _/ |_  )__ / __| |
| _|| / _ \ V  V / |/ / |_ \/ _  |
|_| |_\___/\_/\_/|_/___|___/\__,_|
                         3.0.4
flow@flow:3.0.4 /home/username/local/flow123d_3.0.4$> ls
bin  doc  install.sh  tests uninstall.sh
\end{verbatim}


\subsection{Running Flow123d on Windows}
On system Windows you will have a shortcut on your desktop, to verify everything is working. To run flow123d from anywhere simple type
\verb'flow123d.bat' or \verb'fterm.bat' (in terminal, powershell, Total Commander, \dots).

Each bat file serves a different purpose:
\begin{itemize}
  \item \verb'flow123d.bat' serves as a binary, it possible to run the bat file multiple times (useful for a batch processing)
  \item \verb'fterm.bat' serves as an interactive shell console session invoked inside Docker container.
\end{itemize}

By default \verb'flow123d.bat' will run the last installed version on your system. If you have multiple version installed and want to run specific one, each version has a unique bat files \verb'flow123d-<version>.bat' and \verb'fterm-<version>.bat' file (e.g. \verb'flow123d-3.0.9.bat' and \verb'fterm-3.0.9.bat').


\subsubsection{Running from other batch file}
The Windows system calls the batch files in the different way then the binaries. In particular the calling batch file is not processed further after the child batch
file is done. In order to do so, one have to use the \verb'CALL' command. This is especially necessary for various calibration tools. The correct calling batch file
may look like:
\begin{verbatim}
    echo "Starting Flow123d ..."
    call flow123d.bat a_simulation.yaml
    echo "... simulation done."
\end{verbatim}


\subsubsection{Adjusting memory of virtual machine}
To change the memory limits of the Virtual machine, open the Docker Settings dialog (right click on the whale icon) and select \verb'Settings'.
Navigate to \verb'Advanced' tab and adjust the memory. Detailed instructions can be found \href{https://docs.docker.com/desktop/windows/#advanced}{Docker docs}.


\section{Flow123d arguments}
When you are inside of the Docker container, you have access to the entire file system of the container. Flow123d is installed in \verb'/opt/flow123d' directory. Folder \verb'bin' contains binary files and is automatically
added to the variable \verb'PATH', so that every executable in this folder can be called from anywhere.

The main Flow123d binary is located in \verb'bin/flow123d' and accepts the following arguments:
\begin{description}
 \item[{\tt --help}] \hfill\\
        Help for parameters interpreted by Flow123d. Remaining parameters are passed to PETSC.
 \item[ {\tt -s, --solve} ] \verb'<file>' \hfill\\
 	 Set principal input file. Can be in YAML (or  JSON) file format. All relative paths of the input
 	 files are relative to the location of the principal input file.
 \item[{\tt -i, --input\_dir}] \verb'<directory>' \hfill\\
 	The placeholder \verb"${INPUT}" %$
  	used in the path of an input file will be replaced by the \verb'<directory>'. Default value is \verb'input'.
 \item[{\tt -o, --output\_dir}] \verb'<directory>' \hfill\\
 	All paths for output files will be relative to this \verb'<directory>'. Default value is \verb'output'.
 \item[{\tt -l, --log}] \verb'<file_name>' \hfill\\
 	Set base name of log files. Default value is \verb'flow123d'. The log files are individual for every MPI process, placed in the output directory.
 	The MPI rank of the process and the \verb'log' suffix are appended to the base name.
 \item[{\tt --version}] \hfill\\
        Display version and build information.
 \item[{\tt --no\_log}] \hfill\\
        Turn off logging.
 \item[{\tt --no\_profiler}] \hfill\\
        Turn off profiler output.
 \item[{\tt --petsc\_redirect <file>}] \hfill\\
        Redirect all PETSc stdout and stderr to given file.
 \item[{\tt --input\_format}] \hfill\\
        Print a~description of the main input file in JSON format.
        This is used by GeoMop model editor and by python scripts for
        generating reference documentation in Latex or HTML format.
 \item[{\tt --yaml\_balance}] \hfill\\
        Generate balance file also in machine readable YAML format. Will be default in future, used by GeoMop.
 \item[{\tt --no\_signal\_handler}] \hfill\\
        For debugging purpose.
\end{description}

The {\tt -s} identifier can be skipped:
\begin{verbatim}
  flow123d <main_input>.yaml <options> <PETSC options>
\end{verbatim}

Any different parameters are passed to the PETSC library. An advanced user can influence lot of parameters of linear solvers. In order to get list of supported options
use parameter \verb'-help' together with a~valid input file. Options for various PETSC modules are displayed when the module is used for the first time.

Alternatively, you can use python script \verb'exec_parallel' located in \verb'bin/python' to start parallel jobs or limit resources used by the program.

After double dash specify which \verb'mpiexec' binary will be used (\verb'MPI-EXECUTABLE') and then specify what should be executed.
The script does not need to run solely \verb'flow123d'.

If we want to run command \verb'whoami' in parallel we can do:
\begin{verbatim}
	bin $> exec_parallel -n 4 -- ./mpiexec whoami
\end{verbatim}

To execute Flow123d in parallel we can do:
\begin{verbatim}
	bin $> exec_parallel -n 4 -- ./mpiexec ./flow123d --help
\end{verbatim}


\begin{verbatim}
  exec_parallel [OPTIONS] -- [MPI-EXECUTABLE] [PARAMS]
\end{verbatim}

The script has following options:

\begin{description}
  \item[{\tt -h, --help}] \hfill\\
  	Usage overview.
  \item[{\tt --host}] \verb'<hostname>' \hfill\\
  		Valid only when option \verb'--queue' is set.
        Default value is the host name obtained by python \verb'platform.node()' call, this argument can be used to override it.
        Resulting value is used to select a~correct PBS module from lookup table \verb'config/host_table.yaml'.
  \item[{\tt -n}] \verb'<number of processes>' \hfill\\
  	Specify number of MPI parallel processes for calculation.
  \item[{\tt -t, --limit-time}] \verb'<timeout>' \hfill\\
  	Upper estimate for real running time of the calculation. Kill calculation after {\it timeout} seconds.
  	Value can also be \verb'float' number. When in PBS mode, value can also affect PBS queue.
  \item[{\tt -m, --limit-memory}] \verb'<memory limit>' \hfill\\
  	Limits total available memory to \verb'<memory limit>' MB in total.
  \item[{\tt -q, --queue}] \verb'<queue>' \hfill\\
  		If set activates PBS mode. If argument \verb'queue' is also set selects particular job queue
  		on PBS systems otherwise default PBS queue is used. Default PBS queue automatically
  		choose valid queue based on resources.
\end{description}


Another script which runs Flow123d is \verb'runtest.sh'. This script will run tests specified as arguments. Script accepts both folders
and yaml files. To see full details run \verb'runtest.sh --help'. The script will run yaml tests and then compare results with reference
output. Example usage of the script:

\begin{verbatim}
$> bin/runtest.sh tests/10_darcy/01_source.yaml
configuration:
------------------------------------------------------------
  yaml files       1
  total cases      2
------------------------------------------------------------
execution:
  done [01 of 02] 1 x 10_darcy/01_source [00:03.070] [31 MiB] exitcode 0
  done [02 of 02] 2 x 10_darcy/01_source [00:01.378] [31 MiB] exitcode 0
------------------------------------------------------------
SUMMARY 
------------------------------------------------------------
  [ PASSED ]  | 1 x 10_darcy/01_source                        [ 2.76 sec] 
  [ PASSED ]  | 2 x 10_darcy/01_source                        [ 1.07 sec] 
  ------------------------------------------------------------
  [ PASSED ]  | passed=2, failed=0, skipped=0 in              [ 3.83 sec]
------------------------------------------------------------
\end{verbatim}
